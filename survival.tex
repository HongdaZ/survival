\documentclass{beamer}

\mode<presentation> {
	
	% The Beamer class comes with a number of default slide themes
	% which change the colors and layouts of slides. Below this is a list
	% of all the themes, uncomment each in turn to see what they look like.
	
	%\usetheme{default}
	%\usetheme{AnnArbor}
	%\usetheme{Antibes}
	%\usetheme{Bergen}
	%\usetheme{Berkeley}
	%\usetheme{Berlin}
	%\usetheme{Boadilla}
	%\usetheme{CambridgeUS}
	%\usetheme{Copenhagen}
	%\usetheme{Darmstadt}
	%\usetheme{Dresden}
	%\usetheme{Frankfurt}
	%\usetheme{Goettingen}
	%\usetheme{Hannover}
	%\usetheme{Ilmenau}
	%\usetheme{JuanLesPins}
	%\usetheme{Luebeck}
	\usetheme{Madrid}
	%\usetheme{Malmoe}
	%\usetheme{Marburg}
	%\usetheme{Montpellier}
	%\usetheme{PaloAlto}
	%\usetheme{Pittsburgh}
	%\usetheme{Rochester}
	%\usetheme{Singapore}
	%\usetheme{Szeged}
	%\usetheme{Warsaw}
	
	% As well as themes, the Beamer class has a number of color themes
	% for any slide theme. Uncomment each of these in turn to see how it
	% changes the colors of your current slide theme.
	
	%\usecolortheme{albatross}
	%\usecolortheme{beaver}
	%\usecolortheme{beetle}
	%\usecolortheme{crane}
	%\usecolortheme{dolphin}
	%\usecolortheme{dove}
	%\usecolortheme{fly}
	%\usecolortheme{lily}
	%\usecolortheme{orchid}
	%\usecolortheme{rose}
	%\usecolortheme{seagull}
	%\usecolortheme{seahorse}
	%\usecolortheme{whale}
	%\usecolortheme{wolverine}
	
	%\setbeamertemplate{footline} % To remove the footer line in all slides uncomment this line
	%\setbeamertemplate{footline}[page number] % To replace the footer line in all slides with a simple slide count uncomment this line
	
	%\setbeamertemplate{navigation symbols}{} % To remove the navigation symbols from the bottom of all slides uncomment this line
}

\usepackage{graphicx} % Allows including images
\usepackage{booktabs} % Allows the use of \toprule, \midrule and \bottomrule in tables 

\usepackage[T1]{fontenc}
\usepackage[utf8]{inputenc}
\setbeamertemplate{caption}[numbered]
\newcommand{\C}{\mathbb{C}}
\newcommand{\R}{\mathbb{R}}
\newcommand{\Q}{\mathbb{Q}}
\newcommand{\Z}{\mathbb{Z}}
\newcommand{\N}{\mathbb{N}}
\newcommand{\p}{\mathbb{P}}
\newcommand{\E}{\mathbb{E}}
\usepackage{graphicx}
\usepackage{amssymb}
\usepackage{setspace}
\usepackage[toc,page]{appendix}
\usepackage{epstopdf}
\usepackage{latexsym}
\usepackage{amstext}
\usepackage{lmodern}
\usepackage{amsmath}
\usepackage{bbm}
\usepackage{amsfonts}
\usepackage{url}
\usepackage{bm}
\usepackage{mathrsfs}
\usepackage{mathtools}
\usepackage{float}
%\usepackage{hyperref} give reference hyperlink 
%\usepackage{setspace}
\usepackage{indentfirst}
\usepackage{multirow}
\usepackage{color}
\usepackage{mathtools}
% packages from template
\usepackage{amsmath,amsthm,amssymb,amsfonts}
\usepackage[width=.9\textwidth]{caption}
\usepackage{mathrsfs}
\usepackage{graphicx}
\newcommand{\indep}{\rotatebox[origin=c]{90}{$\models$}}
\usepackage{textgreek}
\usepackage{bbold}
\usepackage{subcaption}
\usepackage{natbib}
\usepackage{verbatim}
\usepackage{soul}
\usepackage[utf8]{inputenc}
\usepackage[algo2e,ruled,vlined]{algorithm2e} 
\usepackage{fancyvrb}


\newtheorem{proposition}[theorem]{Proposition}

\newcommand{\M}{\mathbf{M}}
\newcommand{\rank}{\mathrm{rank}}
\newcommand{\rep}{\mathrm{rep}}
\newcommand{\PR}{\text{Pr}}
\newcommand{\pkg}[1]{{\fontseries{b}\selectfont #1}}
%\newcommand\norm[1]{\left\lVert#1\right\rVert}
\newcommand{\bs}[1]{\boldsymbol{#1}}
\newcommand{\mb}[1]{\mathbf{#1}}
\DeclareMathOperator*{\argmin}{arg\,min}
\DeclarePairedDelimiter{\ceil}{\lceil}{\rceil}
\DeclarePairedDelimiterX{\norm}[1]{\lVert}{\rVert}{#1}
\allowdisplaybreaks
%=============================================================================
% prelude
%=============================================================================
\def\mathLarge#1{\mbox{\LARGE $#1$}}
\usepackage{soul}


\title[]{A Brief Introduction to Survival Analysis}
\author[Hongda Zhang]{Hongda Zhang}
\institute{Nanjing University}
\date{\today}



\begin{document}
	\begin{frame}
		\titlepage
	\end{frame}
	
	\begin{frame}
		\frametitle{Contents}
		\begin{enumerate}
			\item Basic concepts
			\item Kaplan-Meier estimate
			\item Cox regression model
		\end{enumerate}
	\nocite{*}
	\end{frame}

	\begin{frame}
		\frametitle{Basic concepts}
		Survival analysis is a set of statistical methods used to deal with time-to-event problems.
		\begin{itemize}
			\item Survival time (denoted by $t$): the length of time from a starting point to the occurrence of some event of interest. 
			\item Censoring: a survival time is censored if the end-point is not observed. Censored survival time is denoted by $c$.
				\begin{itemize}
					\item right censoring: the actual survival time is greater than that observed.\\
					\item left censoring: the actual survival time is less than that observed.\\
					\item interval censoring: the event happened within an interval.
				\end{itemize}
			\item Independent censoring: the actual survival time is independent from the cause of censoring. Many of the survival analysis methods are not valid without this assumption.
		\end{itemize}
	\end{frame}

	\begin{frame}
		\frametitle{Basic concepts (cont.)}
		\begin{itemize}
			\item The distribution function: let $T$ be the random variable associated with the survival time. Then the distribution of function of $T$ is then given by 
			\[ F( t ) = P( T < t ) = \int_{ 0 }^{ t } f( u ) du,  \]
			which represents the probability of the survival time being less than $t$.
			\item The survivor function:
			\[ S(t) = P( T \geq t ) = 1 - F( t ), \]
			which represents the probability of survival time greater than or equal to $t$.
		\end{itemize}
	\end{frame}

	\begin{frame}
		\frametitle{Basic concepts (cont.)}
		\begin{itemize}
			\item The hazard function: 
			\[ h(t) = \lim_{  \delta \rightarrow 0 }\left\{ \frac{ P( t \leq T < t + \delta | T \geq t ) }{ \delta } \right\}, \]
			which represents a instantaneous event rate at time $t$. 
			(Note: $ h( t ) = \frac{ f( t ) }{ S( t ) }$)
			\item The cumulative hazard function:
			\[ H( t ) = \int_{ 0 }^{ t } h( u ) du, \]
			represents the cumulative risk of an event happening by time $t$.
			(Note: $H( t ) = - \log S( t )$.) 
		\end{itemize}
	\end{frame}

	\begin{frame}[allowframebreaks]
		\begin{singlespace}
			\interlinepenalty=10000	% prevents bib items from splitting across pages
			\bibliography{survival}
			\bibliographystyle{apalike}
		\end{singlespace}
	\end{frame}
\end{document}